\chapter{Data Pre-processing}
\label{chap:preprocessing}
Data pre-preprocessing is one of the first and most important steps in the analysis of any kind. Pre-processing routines aim to enhance data quality, removing noise and unwanted, useless information that distorts the classification process. The specific steps that are to be taken are dependent of the nature of data. The following sections explain the steps taken to perform the initial data pre-processing

\section{Color removal}
\label{sec:colorRemoval}
As seen in figure \ref{fig:sampleCaptcha}, images are in color (presumably to add randomness). To our classification purposes color is irrelevant, because color doesn't have any information that helps us to determine whether a symbol is a number or another. So the first step to take is the color removal. 
\section{Cropping}

\section{Data augmentation}
    \subsection{Rotation}
    \subsection{Translation}